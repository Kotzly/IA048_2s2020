\documentclass[11pt]{article}

    \usepackage[breakable]{tcolorbox}
    \usepackage{parskip} % Stop auto-indenting (to mimic markdown behaviour)
    
    \usepackage{iftex}
    \ifPDFTeX
    	\usepackage[T1]{fontenc}
    	\usepackage{mathpazo}
    \else
    	\usepackage{fontspec}
    \fi

    % Basic figure setup, for now with no caption control since it's done
    % automatically by Pandoc (which extracts ![](path) syntax from Markdown).
    \usepackage{graphicx}
    % Maintain compatibility with old templates. Remove in nbconvert 6.0
    \let\Oldincludegraphics\includegraphics
    % Ensure that by default, figures have no caption (until we provide a
    % proper Figure object with a Caption API and a way to capture that
    % in the conversion process - todo).
    \usepackage{caption}
    \DeclareCaptionFormat{nocaption}{}
    \captionsetup{format=nocaption,aboveskip=0pt,belowskip=0pt}

    \usepackage{float}
    \floatplacement{figure}{H} % forces figures to be placed at the correct location
    \usepackage{xcolor} % Allow colors to be defined
    \usepackage{enumerate} % Needed for markdown enumerations to work
    \usepackage{geometry} % Used to adjust the document margins
    \usepackage{amsmath} % Equations
    \usepackage{amssymb} % Equations
    \usepackage{textcomp} % defines textquotesingle
    % Hack from http://tex.stackexchange.com/a/47451/13684:
    \AtBeginDocument{%
        \def\PYZsq{\textquotesingle}% Upright quotes in Pygmentized code
    }
    \usepackage{upquote} % Upright quotes for verbatim code
    \usepackage{eurosym} % defines \euro
    \usepackage[mathletters]{ucs} % Extended unicode (utf-8) support
    \usepackage{fancyvrb} % verbatim replacement that allows latex
    \usepackage{grffile} % extends the file name processing of package graphics 
                         % to support a larger range
    \makeatletter % fix for old versions of grffile with XeLaTeX
    \@ifpackagelater{grffile}{2019/11/01}
    {
      % Do nothing on new versions
    }
    {
      \def\Gread@@xetex#1{%
        \IfFileExists{"\Gin@base".bb}%
        {\Gread@eps{\Gin@base.bb}}%
        {\Gread@@xetex@aux#1}%
      }
    }
    \makeatother
    \usepackage[Export]{adjustbox} % Used to constrain images to a maximum size
    \adjustboxset{max size={0.9\linewidth}{0.9\paperheight}}

    % The hyperref package gives us a pdf with properly built
    % internal navigation ('pdf bookmarks' for the table of contents,
    % internal cross-reference links, web links for URLs, etc.)
    \usepackage{hyperref}
    % The default LaTeX title has an obnoxious amount of whitespace. By default,
    % titling removes some of it. It also provides customization options.
    \usepackage{titling}
    \usepackage{longtable} % longtable support required by pandoc >1.10
    \usepackage{booktabs}  % table support for pandoc > 1.12.2
    \usepackage[inline]{enumitem} % IRkernel/repr support (it uses the enumerate* environment)
    \usepackage[normalem]{ulem} % ulem is needed to support strikethroughs (\sout)
                                % normalem makes italics be italics, not underlines
    \usepackage{mathrsfs}
    

    
    % Colors for the hyperref package
    \definecolor{urlcolor}{rgb}{0,.145,.698}
    \definecolor{linkcolor}{rgb}{.71,0.21,0.01}
    \definecolor{citecolor}{rgb}{.12,.54,.11}

    % ANSI colors
    \definecolor{ansi-black}{HTML}{3E424D}
    \definecolor{ansi-black-intense}{HTML}{282C36}
    \definecolor{ansi-red}{HTML}{E75C58}
    \definecolor{ansi-red-intense}{HTML}{B22B31}
    \definecolor{ansi-green}{HTML}{00A250}
    \definecolor{ansi-green-intense}{HTML}{007427}
    \definecolor{ansi-yellow}{HTML}{DDB62B}
    \definecolor{ansi-yellow-intense}{HTML}{B27D12}
    \definecolor{ansi-blue}{HTML}{208FFB}
    \definecolor{ansi-blue-intense}{HTML}{0065CA}
    \definecolor{ansi-magenta}{HTML}{D160C4}
    \definecolor{ansi-magenta-intense}{HTML}{A03196}
    \definecolor{ansi-cyan}{HTML}{60C6C8}
    \definecolor{ansi-cyan-intense}{HTML}{258F8F}
    \definecolor{ansi-white}{HTML}{C5C1B4}
    \definecolor{ansi-white-intense}{HTML}{A1A6B2}
    \definecolor{ansi-default-inverse-fg}{HTML}{FFFFFF}
    \definecolor{ansi-default-inverse-bg}{HTML}{000000}

    % common color for the border for error outputs.
    \definecolor{outerrorbackground}{HTML}{FFDFDF}

    % commands and environments needed by pandoc snippets
    % extracted from the output of `pandoc -s`
    \providecommand{\tightlist}{%
      \setlength{\itemsep}{0pt}\setlength{\parskip}{0pt}}
    \DefineVerbatimEnvironment{Highlighting}{Verbatim}{commandchars=\\\{\}}
    % Add ',fontsize=\small' for more characters per line
    \newenvironment{Shaded}{}{}
    \newcommand{\KeywordTok}[1]{\textcolor[rgb]{0.00,0.44,0.13}{\textbf{{#1}}}}
    \newcommand{\DataTypeTok}[1]{\textcolor[rgb]{0.56,0.13,0.00}{{#1}}}
    \newcommand{\DecValTok}[1]{\textcolor[rgb]{0.25,0.63,0.44}{{#1}}}
    \newcommand{\BaseNTok}[1]{\textcolor[rgb]{0.25,0.63,0.44}{{#1}}}
    \newcommand{\FloatTok}[1]{\textcolor[rgb]{0.25,0.63,0.44}{{#1}}}
    \newcommand{\CharTok}[1]{\textcolor[rgb]{0.25,0.44,0.63}{{#1}}}
    \newcommand{\StringTok}[1]{\textcolor[rgb]{0.25,0.44,0.63}{{#1}}}
    \newcommand{\CommentTok}[1]{\textcolor[rgb]{0.38,0.63,0.69}{\textit{{#1}}}}
    \newcommand{\OtherTok}[1]{\textcolor[rgb]{0.00,0.44,0.13}{{#1}}}
    \newcommand{\AlertTok}[1]{\textcolor[rgb]{1.00,0.00,0.00}{\textbf{{#1}}}}
    \newcommand{\FunctionTok}[1]{\textcolor[rgb]{0.02,0.16,0.49}{{#1}}}
    \newcommand{\RegionMarkerTok}[1]{{#1}}
    \newcommand{\ErrorTok}[1]{\textcolor[rgb]{1.00,0.00,0.00}{\textbf{{#1}}}}
    \newcommand{\NormalTok}[1]{{#1}}
    
    % Additional commands for more recent versions of Pandoc
    \newcommand{\ConstantTok}[1]{\textcolor[rgb]{0.53,0.00,0.00}{{#1}}}
    \newcommand{\SpecialCharTok}[1]{\textcolor[rgb]{0.25,0.44,0.63}{{#1}}}
    \newcommand{\VerbatimStringTok}[1]{\textcolor[rgb]{0.25,0.44,0.63}{{#1}}}
    \newcommand{\SpecialStringTok}[1]{\textcolor[rgb]{0.73,0.40,0.53}{{#1}}}
    \newcommand{\ImportTok}[1]{{#1}}
    \newcommand{\DocumentationTok}[1]{\textcolor[rgb]{0.73,0.13,0.13}{\textit{{#1}}}}
    \newcommand{\AnnotationTok}[1]{\textcolor[rgb]{0.38,0.63,0.69}{\textbf{\textit{{#1}}}}}
    \newcommand{\CommentVarTok}[1]{\textcolor[rgb]{0.38,0.63,0.69}{\textbf{\textit{{#1}}}}}
    \newcommand{\VariableTok}[1]{\textcolor[rgb]{0.10,0.09,0.49}{{#1}}}
    \newcommand{\ControlFlowTok}[1]{\textcolor[rgb]{0.00,0.44,0.13}{\textbf{{#1}}}}
    \newcommand{\OperatorTok}[1]{\textcolor[rgb]{0.40,0.40,0.40}{{#1}}}
    \newcommand{\BuiltInTok}[1]{{#1}}
    \newcommand{\ExtensionTok}[1]{{#1}}
    \newcommand{\PreprocessorTok}[1]{\textcolor[rgb]{0.74,0.48,0.00}{{#1}}}
    \newcommand{\AttributeTok}[1]{\textcolor[rgb]{0.49,0.56,0.16}{{#1}}}
    \newcommand{\InformationTok}[1]{\textcolor[rgb]{0.38,0.63,0.69}{\textbf{\textit{{#1}}}}}
    \newcommand{\WarningTok}[1]{\textcolor[rgb]{0.38,0.63,0.69}{\textbf{\textit{{#1}}}}}
    
    
    % Define a nice break command that doesn't care if a line doesn't already
    % exist.
    \def\br{\hspace*{\fill} \\* }
    % Math Jax compatibility definitions
    \def\gt{>}
    \def\lt{<}
    \let\Oldtex\TeX
    \let\Oldlatex\LaTeX
    \renewcommand{\TeX}{\textrm{\Oldtex}}
    \renewcommand{\LaTeX}{\textrm{\Oldlatex}}
    % Document parameters
    % Document title
    \title{EFC3\_Pt1}
    
    
    
    
    
% Pygments definitions
\makeatletter
\def\PY@reset{\let\PY@it=\relax \let\PY@bf=\relax%
    \let\PY@ul=\relax \let\PY@tc=\relax%
    \let\PY@bc=\relax \let\PY@ff=\relax}
\def\PY@tok#1{\csname PY@tok@#1\endcsname}
\def\PY@toks#1+{\ifx\relax#1\empty\else%
    \PY@tok{#1}\expandafter\PY@toks\fi}
\def\PY@do#1{\PY@bc{\PY@tc{\PY@ul{%
    \PY@it{\PY@bf{\PY@ff{#1}}}}}}}
\def\PY#1#2{\PY@reset\PY@toks#1+\relax+\PY@do{#2}}

\expandafter\def\csname PY@tok@w\endcsname{\def\PY@tc##1{\textcolor[rgb]{0.73,0.73,0.73}{##1}}}
\expandafter\def\csname PY@tok@c\endcsname{\let\PY@it=\textit\def\PY@tc##1{\textcolor[rgb]{0.25,0.50,0.50}{##1}}}
\expandafter\def\csname PY@tok@cp\endcsname{\def\PY@tc##1{\textcolor[rgb]{0.74,0.48,0.00}{##1}}}
\expandafter\def\csname PY@tok@k\endcsname{\let\PY@bf=\textbf\def\PY@tc##1{\textcolor[rgb]{0.00,0.50,0.00}{##1}}}
\expandafter\def\csname PY@tok@kp\endcsname{\def\PY@tc##1{\textcolor[rgb]{0.00,0.50,0.00}{##1}}}
\expandafter\def\csname PY@tok@kt\endcsname{\def\PY@tc##1{\textcolor[rgb]{0.69,0.00,0.25}{##1}}}
\expandafter\def\csname PY@tok@o\endcsname{\def\PY@tc##1{\textcolor[rgb]{0.40,0.40,0.40}{##1}}}
\expandafter\def\csname PY@tok@ow\endcsname{\let\PY@bf=\textbf\def\PY@tc##1{\textcolor[rgb]{0.67,0.13,1.00}{##1}}}
\expandafter\def\csname PY@tok@nb\endcsname{\def\PY@tc##1{\textcolor[rgb]{0.00,0.50,0.00}{##1}}}
\expandafter\def\csname PY@tok@nf\endcsname{\def\PY@tc##1{\textcolor[rgb]{0.00,0.00,1.00}{##1}}}
\expandafter\def\csname PY@tok@nc\endcsname{\let\PY@bf=\textbf\def\PY@tc##1{\textcolor[rgb]{0.00,0.00,1.00}{##1}}}
\expandafter\def\csname PY@tok@nn\endcsname{\let\PY@bf=\textbf\def\PY@tc##1{\textcolor[rgb]{0.00,0.00,1.00}{##1}}}
\expandafter\def\csname PY@tok@ne\endcsname{\let\PY@bf=\textbf\def\PY@tc##1{\textcolor[rgb]{0.82,0.25,0.23}{##1}}}
\expandafter\def\csname PY@tok@nv\endcsname{\def\PY@tc##1{\textcolor[rgb]{0.10,0.09,0.49}{##1}}}
\expandafter\def\csname PY@tok@no\endcsname{\def\PY@tc##1{\textcolor[rgb]{0.53,0.00,0.00}{##1}}}
\expandafter\def\csname PY@tok@nl\endcsname{\def\PY@tc##1{\textcolor[rgb]{0.63,0.63,0.00}{##1}}}
\expandafter\def\csname PY@tok@ni\endcsname{\let\PY@bf=\textbf\def\PY@tc##1{\textcolor[rgb]{0.60,0.60,0.60}{##1}}}
\expandafter\def\csname PY@tok@na\endcsname{\def\PY@tc##1{\textcolor[rgb]{0.49,0.56,0.16}{##1}}}
\expandafter\def\csname PY@tok@nt\endcsname{\let\PY@bf=\textbf\def\PY@tc##1{\textcolor[rgb]{0.00,0.50,0.00}{##1}}}
\expandafter\def\csname PY@tok@nd\endcsname{\def\PY@tc##1{\textcolor[rgb]{0.67,0.13,1.00}{##1}}}
\expandafter\def\csname PY@tok@s\endcsname{\def\PY@tc##1{\textcolor[rgb]{0.73,0.13,0.13}{##1}}}
\expandafter\def\csname PY@tok@sd\endcsname{\let\PY@it=\textit\def\PY@tc##1{\textcolor[rgb]{0.73,0.13,0.13}{##1}}}
\expandafter\def\csname PY@tok@si\endcsname{\let\PY@bf=\textbf\def\PY@tc##1{\textcolor[rgb]{0.73,0.40,0.53}{##1}}}
\expandafter\def\csname PY@tok@se\endcsname{\let\PY@bf=\textbf\def\PY@tc##1{\textcolor[rgb]{0.73,0.40,0.13}{##1}}}
\expandafter\def\csname PY@tok@sr\endcsname{\def\PY@tc##1{\textcolor[rgb]{0.73,0.40,0.53}{##1}}}
\expandafter\def\csname PY@tok@ss\endcsname{\def\PY@tc##1{\textcolor[rgb]{0.10,0.09,0.49}{##1}}}
\expandafter\def\csname PY@tok@sx\endcsname{\def\PY@tc##1{\textcolor[rgb]{0.00,0.50,0.00}{##1}}}
\expandafter\def\csname PY@tok@m\endcsname{\def\PY@tc##1{\textcolor[rgb]{0.40,0.40,0.40}{##1}}}
\expandafter\def\csname PY@tok@gh\endcsname{\let\PY@bf=\textbf\def\PY@tc##1{\textcolor[rgb]{0.00,0.00,0.50}{##1}}}
\expandafter\def\csname PY@tok@gu\endcsname{\let\PY@bf=\textbf\def\PY@tc##1{\textcolor[rgb]{0.50,0.00,0.50}{##1}}}
\expandafter\def\csname PY@tok@gd\endcsname{\def\PY@tc##1{\textcolor[rgb]{0.63,0.00,0.00}{##1}}}
\expandafter\def\csname PY@tok@gi\endcsname{\def\PY@tc##1{\textcolor[rgb]{0.00,0.63,0.00}{##1}}}
\expandafter\def\csname PY@tok@gr\endcsname{\def\PY@tc##1{\textcolor[rgb]{1.00,0.00,0.00}{##1}}}
\expandafter\def\csname PY@tok@ge\endcsname{\let\PY@it=\textit}
\expandafter\def\csname PY@tok@gs\endcsname{\let\PY@bf=\textbf}
\expandafter\def\csname PY@tok@gp\endcsname{\let\PY@bf=\textbf\def\PY@tc##1{\textcolor[rgb]{0.00,0.00,0.50}{##1}}}
\expandafter\def\csname PY@tok@go\endcsname{\def\PY@tc##1{\textcolor[rgb]{0.53,0.53,0.53}{##1}}}
\expandafter\def\csname PY@tok@gt\endcsname{\def\PY@tc##1{\textcolor[rgb]{0.00,0.27,0.87}{##1}}}
\expandafter\def\csname PY@tok@err\endcsname{\def\PY@bc##1{\setlength{\fboxsep}{0pt}\fcolorbox[rgb]{1.00,0.00,0.00}{1,1,1}{\strut ##1}}}
\expandafter\def\csname PY@tok@kc\endcsname{\let\PY@bf=\textbf\def\PY@tc##1{\textcolor[rgb]{0.00,0.50,0.00}{##1}}}
\expandafter\def\csname PY@tok@kd\endcsname{\let\PY@bf=\textbf\def\PY@tc##1{\textcolor[rgb]{0.00,0.50,0.00}{##1}}}
\expandafter\def\csname PY@tok@kn\endcsname{\let\PY@bf=\textbf\def\PY@tc##1{\textcolor[rgb]{0.00,0.50,0.00}{##1}}}
\expandafter\def\csname PY@tok@kr\endcsname{\let\PY@bf=\textbf\def\PY@tc##1{\textcolor[rgb]{0.00,0.50,0.00}{##1}}}
\expandafter\def\csname PY@tok@bp\endcsname{\def\PY@tc##1{\textcolor[rgb]{0.00,0.50,0.00}{##1}}}
\expandafter\def\csname PY@tok@fm\endcsname{\def\PY@tc##1{\textcolor[rgb]{0.00,0.00,1.00}{##1}}}
\expandafter\def\csname PY@tok@vc\endcsname{\def\PY@tc##1{\textcolor[rgb]{0.10,0.09,0.49}{##1}}}
\expandafter\def\csname PY@tok@vg\endcsname{\def\PY@tc##1{\textcolor[rgb]{0.10,0.09,0.49}{##1}}}
\expandafter\def\csname PY@tok@vi\endcsname{\def\PY@tc##1{\textcolor[rgb]{0.10,0.09,0.49}{##1}}}
\expandafter\def\csname PY@tok@vm\endcsname{\def\PY@tc##1{\textcolor[rgb]{0.10,0.09,0.49}{##1}}}
\expandafter\def\csname PY@tok@sa\endcsname{\def\PY@tc##1{\textcolor[rgb]{0.73,0.13,0.13}{##1}}}
\expandafter\def\csname PY@tok@sb\endcsname{\def\PY@tc##1{\textcolor[rgb]{0.73,0.13,0.13}{##1}}}
\expandafter\def\csname PY@tok@sc\endcsname{\def\PY@tc##1{\textcolor[rgb]{0.73,0.13,0.13}{##1}}}
\expandafter\def\csname PY@tok@dl\endcsname{\def\PY@tc##1{\textcolor[rgb]{0.73,0.13,0.13}{##1}}}
\expandafter\def\csname PY@tok@s2\endcsname{\def\PY@tc##1{\textcolor[rgb]{0.73,0.13,0.13}{##1}}}
\expandafter\def\csname PY@tok@sh\endcsname{\def\PY@tc##1{\textcolor[rgb]{0.73,0.13,0.13}{##1}}}
\expandafter\def\csname PY@tok@s1\endcsname{\def\PY@tc##1{\textcolor[rgb]{0.73,0.13,0.13}{##1}}}
\expandafter\def\csname PY@tok@mb\endcsname{\def\PY@tc##1{\textcolor[rgb]{0.40,0.40,0.40}{##1}}}
\expandafter\def\csname PY@tok@mf\endcsname{\def\PY@tc##1{\textcolor[rgb]{0.40,0.40,0.40}{##1}}}
\expandafter\def\csname PY@tok@mh\endcsname{\def\PY@tc##1{\textcolor[rgb]{0.40,0.40,0.40}{##1}}}
\expandafter\def\csname PY@tok@mi\endcsname{\def\PY@tc##1{\textcolor[rgb]{0.40,0.40,0.40}{##1}}}
\expandafter\def\csname PY@tok@il\endcsname{\def\PY@tc##1{\textcolor[rgb]{0.40,0.40,0.40}{##1}}}
\expandafter\def\csname PY@tok@mo\endcsname{\def\PY@tc##1{\textcolor[rgb]{0.40,0.40,0.40}{##1}}}
\expandafter\def\csname PY@tok@ch\endcsname{\let\PY@it=\textit\def\PY@tc##1{\textcolor[rgb]{0.25,0.50,0.50}{##1}}}
\expandafter\def\csname PY@tok@cm\endcsname{\let\PY@it=\textit\def\PY@tc##1{\textcolor[rgb]{0.25,0.50,0.50}{##1}}}
\expandafter\def\csname PY@tok@cpf\endcsname{\let\PY@it=\textit\def\PY@tc##1{\textcolor[rgb]{0.25,0.50,0.50}{##1}}}
\expandafter\def\csname PY@tok@c1\endcsname{\let\PY@it=\textit\def\PY@tc##1{\textcolor[rgb]{0.25,0.50,0.50}{##1}}}
\expandafter\def\csname PY@tok@cs\endcsname{\let\PY@it=\textit\def\PY@tc##1{\textcolor[rgb]{0.25,0.50,0.50}{##1}}}

\def\PYZbs{\char`\\}
\def\PYZus{\char`\_}
\def\PYZob{\char`\{}
\def\PYZcb{\char`\}}
\def\PYZca{\char`\^}
\def\PYZam{\char`\&}
\def\PYZlt{\char`\<}
\def\PYZgt{\char`\>}
\def\PYZsh{\char`\#}
\def\PYZpc{\char`\%}
\def\PYZdl{\char`\$}
\def\PYZhy{\char`\-}
\def\PYZsq{\char`\'}
\def\PYZdq{\char`\"}
\def\PYZti{\char`\~}
% for compatibility with earlier versions
\def\PYZat{@}
\def\PYZlb{[}
\def\PYZrb{]}
\makeatother


    % For linebreaks inside Verbatim environment from package fancyvrb. 
    \makeatletter
        \newbox\Wrappedcontinuationbox 
        \newbox\Wrappedvisiblespacebox 
        \newcommand*\Wrappedvisiblespace {\textcolor{red}{\textvisiblespace}} 
        \newcommand*\Wrappedcontinuationsymbol {\textcolor{red}{\llap{\tiny$\m@th\hookrightarrow$}}} 
        \newcommand*\Wrappedcontinuationindent {3ex } 
        \newcommand*\Wrappedafterbreak {\kern\Wrappedcontinuationindent\copy\Wrappedcontinuationbox} 
        % Take advantage of the already applied Pygments mark-up to insert 
        % potential linebreaks for TeX processing. 
        %        {, <, #, %, $, ' and ": go to next line. 
        %        _, }, ^, &, >, - and ~: stay at end of broken line. 
        % Use of \textquotesingle for straight quote. 
        \newcommand*\Wrappedbreaksatspecials {% 
            \def\PYGZus{\discretionary{\char`\_}{\Wrappedafterbreak}{\char`\_}}% 
            \def\PYGZob{\discretionary{}{\Wrappedafterbreak\char`\{}{\char`\{}}% 
            \def\PYGZcb{\discretionary{\char`\}}{\Wrappedafterbreak}{\char`\}}}% 
            \def\PYGZca{\discretionary{\char`\^}{\Wrappedafterbreak}{\char`\^}}% 
            \def\PYGZam{\discretionary{\char`\&}{\Wrappedafterbreak}{\char`\&}}% 
            \def\PYGZlt{\discretionary{}{\Wrappedafterbreak\char`\<}{\char`\<}}% 
            \def\PYGZgt{\discretionary{\char`\>}{\Wrappedafterbreak}{\char`\>}}% 
            \def\PYGZsh{\discretionary{}{\Wrappedafterbreak\char`\#}{\char`\#}}% 
            \def\PYGZpc{\discretionary{}{\Wrappedafterbreak\char`\%}{\char`\%}}% 
            \def\PYGZdl{\discretionary{}{\Wrappedafterbreak\char`\$}{\char`\$}}% 
            \def\PYGZhy{\discretionary{\char`\-}{\Wrappedafterbreak}{\char`\-}}% 
            \def\PYGZsq{\discretionary{}{\Wrappedafterbreak\textquotesingle}{\textquotesingle}}% 
            \def\PYGZdq{\discretionary{}{\Wrappedafterbreak\char`\"}{\char`\"}}% 
            \def\PYGZti{\discretionary{\char`\~}{\Wrappedafterbreak}{\char`\~}}% 
        } 
        % Some characters . , ; ? ! / are not pygmentized. 
        % This macro makes them "active" and they will insert potential linebreaks 
        \newcommand*\Wrappedbreaksatpunct {% 
            \lccode`\~`\.\lowercase{\def~}{\discretionary{\hbox{\char`\.}}{\Wrappedafterbreak}{\hbox{\char`\.}}}% 
            \lccode`\~`\,\lowercase{\def~}{\discretionary{\hbox{\char`\,}}{\Wrappedafterbreak}{\hbox{\char`\,}}}% 
            \lccode`\~`\;\lowercase{\def~}{\discretionary{\hbox{\char`\;}}{\Wrappedafterbreak}{\hbox{\char`\;}}}% 
            \lccode`\~`\:\lowercase{\def~}{\discretionary{\hbox{\char`\:}}{\Wrappedafterbreak}{\hbox{\char`\:}}}% 
            \lccode`\~`\?\lowercase{\def~}{\discretionary{\hbox{\char`\?}}{\Wrappedafterbreak}{\hbox{\char`\?}}}% 
            \lccode`\~`\!\lowercase{\def~}{\discretionary{\hbox{\char`\!}}{\Wrappedafterbreak}{\hbox{\char`\!}}}% 
            \lccode`\~`\/\lowercase{\def~}{\discretionary{\hbox{\char`\/}}{\Wrappedafterbreak}{\hbox{\char`\/}}}% 
            \catcode`\.\active
            \catcode`\,\active 
            \catcode`\;\active
            \catcode`\:\active
            \catcode`\?\active
            \catcode`\!\active
            \catcode`\/\active 
            \lccode`\~`\~ 	
        }
    \makeatother

    \let\OriginalVerbatim=\Verbatim
    \makeatletter
    \renewcommand{\Verbatim}[1][1]{%
        %\parskip\z@skip
        \sbox\Wrappedcontinuationbox {\Wrappedcontinuationsymbol}%
        \sbox\Wrappedvisiblespacebox {\FV@SetupFont\Wrappedvisiblespace}%
        \def\FancyVerbFormatLine ##1{\hsize\linewidth
            \vtop{\raggedright\hyphenpenalty\z@\exhyphenpenalty\z@
                \doublehyphendemerits\z@\finalhyphendemerits\z@
                \strut ##1\strut}%
        }%
        % If the linebreak is at a space, the latter will be displayed as visible
        % space at end of first line, and a continuation symbol starts next line.
        % Stretch/shrink are however usually zero for typewriter font.
        \def\FV@Space {%
            \nobreak\hskip\z@ plus\fontdimen3\font minus\fontdimen4\font
            \discretionary{\copy\Wrappedvisiblespacebox}{\Wrappedafterbreak}
            {\kern\fontdimen2\font}%
        }%
        
        % Allow breaks at special characters using \PYG... macros.
        \Wrappedbreaksatspecials
        % Breaks at punctuation characters . , ; ? ! and / need catcode=\active 	
        \OriginalVerbatim[#1,codes*=\Wrappedbreaksatpunct]%
    }
    \makeatother

    % Exact colors from NB
    \definecolor{incolor}{HTML}{303F9F}
    \definecolor{outcolor}{HTML}{D84315}
    \definecolor{cellborder}{HTML}{CFCFCF}
    \definecolor{cellbackground}{HTML}{F7F7F7}
    
    % prompt
    \makeatletter
    \newcommand{\boxspacing}{\kern\kvtcb@left@rule\kern\kvtcb@boxsep}
    \makeatother
    \newcommand{\prompt}[4]{
        {\ttfamily\llap{{\color{#2}[#3]:\hspace{3pt}#4}}\vspace{-\baselineskip}}
    }
    

    
    % Prevent overflowing lines due to hard-to-break entities
    \sloppy 
    % Setup hyperref package
    \hypersetup{
      breaklinks=true,  % so long urls are correctly broken across lines
      colorlinks=true,
      urlcolor=urlcolor,
      linkcolor=linkcolor,
      citecolor=citecolor,
      }
    % Slightly bigger margins than the latex defaults
    
    \geometry{verbose,tmargin=1in,bmargin=1in,lmargin=1in,rmargin=1in}
    
    

\begin{document}
    
    \maketitle
    
    

    
    \begin{tcolorbox}[breakable, size=fbox, boxrule=1pt, pad at break*=1mm,colback=cellbackground, colframe=cellborder]
\prompt{In}{incolor}{1}{\boxspacing}
\begin{Verbatim}[commandchars=\\\{\}]
\PY{k+kn}{import} \PY{n+nn}{pandas} \PY{k}{as} \PY{n+nn}{pd}
\PY{k+kn}{import} \PY{n+nn}{numpy} \PY{k}{as} \PY{n+nn}{np}
\PY{k+kn}{import} \PY{n+nn}{matplotlib}\PY{n+nn}{.}\PY{n+nn}{pyplot} \PY{k}{as} \PY{n+nn}{plt}
\PY{k+kn}{import} \PY{n+nn}{tensorflow} \PY{k}{as} \PY{n+nn}{tf}
\PY{k+kn}{import} \PY{n+nn}{warnings}
\PY{k+kn}{from} \PY{n+nn}{tqdm} \PY{k+kn}{import} \PY{n}{tqdm}

\PY{k+kn}{from} \PY{n+nn}{sklearn}\PY{n+nn}{.}\PY{n+nn}{compose} \PY{k+kn}{import} \PY{n}{ColumnTransformer}
\PY{k+kn}{from} \PY{n+nn}{sklearn}\PY{n+nn}{.}\PY{n+nn}{preprocessing} \PY{k+kn}{import} \PY{n}{QuantileTransformer}\PY{p}{,} \PY{n}{RobustScaler}\PY{p}{,} \PY{n}{MinMaxScaler}
\PY{k+kn}{from} \PY{n+nn}{sklearn}\PY{n+nn}{.}\PY{n+nn}{model\PYZus{}selection} \PY{k+kn}{import} \PY{n}{train\PYZus{}test\PYZus{}split}
\PY{k+kn}{from} \PY{n+nn}{sklearn}\PY{n+nn}{.}\PY{n+nn}{model\PYZus{}selection} \PY{k+kn}{import} \PY{n}{StratifiedKFold}
\PY{k+kn}{from} \PY{n+nn}{sklearn}\PY{n+nn}{.}\PY{n+nn}{metrics} \PY{k+kn}{import} \PY{n}{f1\PYZus{}score}\PY{p}{,} \PY{n}{precision\PYZus{}score}\PY{p}{,} \PY{n}{recall\PYZus{}score}\PY{p}{,} \PY{n}{balanced\PYZus{}accuracy\PYZus{}score}
\PY{k+kn}{from} \PY{n+nn}{sklearn}\PY{n+nn}{.}\PY{n+nn}{neural\PYZus{}network} \PY{k+kn}{import} \PY{n}{MLPClassifier}
\PY{k+kn}{from} \PY{n+nn}{sklearn}\PY{n+nn}{.}\PY{n+nn}{metrics} \PY{k+kn}{import} \PY{n}{classification\PYZus{}report}\PY{p}{,} \PY{n}{confusion\PYZus{}matrix}\PY{p}{,} \PY{n}{accuracy\PYZus{}score}

\PY{k+kn}{from} \PY{n+nn}{tensorflow}\PY{n+nn}{.}\PY{n+nn}{keras}\PY{n+nn}{.}\PY{n+nn}{callbacks} \PY{k+kn}{import} \PY{n}{EarlyStopping}

\PY{n}{warnings}\PY{o}{.}\PY{n}{simplefilter}\PY{p}{(}\PY{l+s+s2}{\PYZdq{}}\PY{l+s+s2}{ignore}\PY{l+s+s2}{\PYZdq{}}\PY{p}{)}

\PY{o}{\PYZpc{}}\PY{k}{matplotlib} inline
\end{Verbatim}
\end{tcolorbox}

    \hypertarget{exercuxedcios-de-fixauxe7uxe3o-de-conceitos-efc-3-2s2020}{%
\section{Exercícios de Fixação de Conceitos (EFC) 3 --
2s2020}\label{exercuxedcios-de-fixauxe7uxe3o-de-conceitos-efc-3-2s2020}}

    \hypertarget{parte-1-classificauxe7uxe3o-binuxe1ria-com-redes-mlp}{%
\subsection{Parte 1 -- Classificação binária com redes
MLP}\label{parte-1-classificauxe7uxe3o-binuxe1ria-com-redes-mlp}}

O problema é de classificação de pessoas como tendo diabetes ou não
baseando-se em algumas características, como idade, pressão sanguínea,
BMI, etc.

    \begin{tcolorbox}[breakable, size=fbox, boxrule=1pt, pad at break*=1mm,colback=cellbackground, colframe=cellborder]
\prompt{In}{incolor}{2}{\boxspacing}
\begin{Verbatim}[commandchars=\\\{\}]
\PY{n}{SEED} \PY{o}{=} \PY{l+m+mi}{100}
\PY{n}{tf}\PY{o}{.}\PY{n}{random}\PY{o}{.}\PY{n}{set\PYZus{}seed}\PY{p}{(}\PY{n}{SEED}\PY{p}{)}
\PY{n}{np}\PY{o}{.}\PY{n}{random}\PY{o}{.}\PY{n}{seed}\PY{p}{(}\PY{n}{SEED}\PY{p}{)}
\end{Verbatim}
\end{tcolorbox}

    A tabela abaixo mostra algumas estatísticas das features. Neste dataset
existem pessoas de 21 à 81 anos, de pessoas que tiveram até 17
gravidezes, e com um intervalo variado de BMI.

    \begin{tcolorbox}[breakable, size=fbox, boxrule=1pt, pad at break*=1mm,colback=cellbackground, colframe=cellborder]
\prompt{In}{incolor}{3}{\boxspacing}
\begin{Verbatim}[commandchars=\\\{\}]
\PY{n}{dataframe} \PY{o}{=} \PY{n}{pd}\PY{o}{.}\PY{n}{read\PYZus{}csv}\PY{p}{(}\PY{l+s+s2}{\PYZdq{}}\PY{l+s+s2}{dados\PYZus{}diabetes.csv}\PY{l+s+s2}{\PYZdq{}}\PY{p}{)}
\PY{n+nb}{print}\PY{p}{(}\PY{n}{dataframe}\PY{o}{.}\PY{n}{columns}\PY{p}{)}
\PY{n}{dataframe}\PY{o}{.}\PY{n}{describe}\PY{p}{(}\PY{p}{)}
\end{Verbatim}
\end{tcolorbox}

    \begin{Verbatim}[commandchars=\\\{\}]
Index(['Pregnancies', 'Glucose', 'BloodPressure', 'SkinThickness', 'Insulin',
       'BMI', 'DiabetesPedigreeFunction', 'Age', 'Outcome'],
      dtype='object')
    \end{Verbatim}

            \begin{tcolorbox}[breakable, size=fbox, boxrule=.5pt, pad at break*=1mm, opacityfill=0]
\prompt{Out}{outcolor}{3}{\boxspacing}
\begin{Verbatim}[commandchars=\\\{\}]
       Pregnancies     Glucose  BloodPressure  SkinThickness     Insulin  \textbackslash{}
count   768.000000  768.000000     768.000000     768.000000  768.000000
mean      3.845052  121.677083      72.389323      29.089844  141.753906
std       3.369578   30.464161      12.106039       8.890820   89.100847
min       0.000000   44.000000      24.000000       7.000000   14.000000
25\%       1.000000   99.750000      64.000000      25.000000  102.500000
50\%       3.000000  117.000000      72.000000      28.000000  102.500000
75\%       6.000000  140.250000      80.000000      32.000000  169.500000
max      17.000000  199.000000     122.000000      99.000000  846.000000

              BMI         Age     Outcome
count  768.000000  768.000000  768.000000
mean    32.434635   33.240885    0.348958
std      6.880498   11.760232    0.476951
min     18.200000   21.000000    0.000000
25\%     27.500000   24.000000    0.000000
50\%     32.050000   29.000000    0.000000
75\%     36.600000   41.000000    1.000000
max     67.100000   81.000000    1.000000
\end{Verbatim}
\end{tcolorbox}
        
    \hypertarget{anuxe1lise-das-features}{%
\section{Análise das features}\label{anuxe1lise-das-features}}

Como pode-se ver nos histogramas a seguir, quase todas as features tem
distribuição normal ou uma distribuição semelhante à distribuição de
poisson (como o número de gravidezes e a idade). Nota-se que a feature
``DiabetesPedigreeFunction'' não aparece aqui, isto acontece porque ela
está em um formato diferente das demais features.

    \begin{tcolorbox}[breakable, size=fbox, boxrule=1pt, pad at break*=1mm,colback=cellbackground, colframe=cellborder]
\prompt{In}{incolor}{4}{\boxspacing}
\begin{Verbatim}[commandchars=\\\{\}]
\PY{n}{plt}\PY{o}{.}\PY{n}{figure}\PY{p}{(}\PY{n}{figsize}\PY{o}{=}\PY{p}{(}\PY{l+m+mi}{12}\PY{p}{,} \PY{l+m+mi}{8}\PY{p}{)}\PY{p}{)}
\PY{n}{dataframe}\PY{o}{.}\PY{n}{hist}\PY{p}{(}\PY{n}{bins}\PY{o}{=}\PY{l+m+mi}{10}\PY{p}{,} \PY{n}{ax}\PY{o}{=}\PY{n}{plt}\PY{o}{.}\PY{n}{gca}\PY{p}{(}\PY{p}{)}\PY{p}{)}\PY{p}{;}
\PY{n}{plt}\PY{o}{.}\PY{n}{tight\PYZus{}layout}\PY{p}{(}\PY{p}{)}
\end{Verbatim}
\end{tcolorbox}

    \begin{center}
    \adjustimage{max size={0.9\linewidth}{0.9\paperheight}}{EFC3_Pt1_files/EFC3_Pt1_7_0.png}
    \end{center}
    { \hspace*{\fill} \\}
    
    Pode-se verificar se a feature de DiabetesPedigreeFunction é categórica:

    \begin{tcolorbox}[breakable, size=fbox, boxrule=1pt, pad at break*=1mm,colback=cellbackground, colframe=cellborder]
\prompt{In}{incolor}{5}{\boxspacing}
\begin{Verbatim}[commandchars=\\\{\}]
\PY{n}{dataframe}\PY{o}{.}\PY{n}{DiabetesPedigreeFunction}\PY{o}{.}\PY{n}{nunique}\PY{p}{(}\PY{p}{)}
\end{Verbatim}
\end{tcolorbox}

            \begin{tcolorbox}[breakable, size=fbox, boxrule=.5pt, pad at break*=1mm, opacityfill=0]
\prompt{Out}{outcolor}{5}{\boxspacing}
\begin{Verbatim}[commandchars=\\\{\}]
517
\end{Verbatim}
\end{tcolorbox}
        
    Conclui-se que ela não é, pois existem valores únicos demais para que
ela possa ser considerada categórica nesta situação. Desta maneira ela
será transformada no número que representa.

    Transformando a feature ``DiabetesPedigreeFunction'' em valores
continuos, obtemos o seguinte histograma:

    \begin{tcolorbox}[breakable, size=fbox, boxrule=1pt, pad at break*=1mm,colback=cellbackground, colframe=cellborder]
\prompt{In}{incolor}{6}{\boxspacing}
\begin{Verbatim}[commandchars=\\\{\}]
\PY{n}{dataframe}\PY{o}{.}\PY{n}{DiabetesPedigreeFunction} \PY{o}{=} \PY{n}{dataframe}\PY{o}{.}\PY{n}{DiabetesPedigreeFunction}\PY{o}{.}\PY{n}{apply}\PY{p}{(}\PY{k}{lambda} \PY{n}{x}\PY{p}{:} \PY{n}{x}\PY{o}{.}\PY{n}{replace}\PY{p}{(}\PY{l+s+s2}{\PYZdq{}}\PY{l+s+s2}{.}\PY{l+s+s2}{\PYZdq{}}\PY{p}{,} \PY{l+s+s2}{\PYZdq{}}\PY{l+s+s2}{\PYZdq{}}\PY{p}{)}\PY{p}{)}\PY{o}{.}\PY{n}{apply}\PY{p}{(}\PY{n+nb}{float}\PY{p}{)}
\PY{n}{dataframe}\PY{o}{.}\PY{n}{DiabetesPedigreeFunction}\PY{o}{.}\PY{n}{hist}\PY{p}{(}\PY{n}{bins}\PY{o}{=}\PY{l+m+mi}{10}\PY{p}{)}
\end{Verbatim}
\end{tcolorbox}

            \begin{tcolorbox}[breakable, size=fbox, boxrule=.5pt, pad at break*=1mm, opacityfill=0]
\prompt{Out}{outcolor}{6}{\boxspacing}
\begin{Verbatim}[commandchars=\\\{\}]
<AxesSubplot:>
\end{Verbatim}
\end{tcolorbox}
        
    \begin{center}
    \adjustimage{max size={0.9\linewidth}{0.9\paperheight}}{EFC3_Pt1_files/EFC3_Pt1_12_1.png}
    \end{center}
    { \hspace*{\fill} \\}
    
    No histograma viu-se que o atributo ``Outcome'' não está balanceado. Na
verdade este dado é a variável que iremos querer predizer através de
todos os outros atributos. A distribuição das classes é:

    \begin{tcolorbox}[breakable, size=fbox, boxrule=1pt, pad at break*=1mm,colback=cellbackground, colframe=cellborder]
\prompt{In}{incolor}{7}{\boxspacing}
\begin{Verbatim}[commandchars=\\\{\}]
\PY{n}{classes}\PY{p}{,} \PY{n}{counts} \PY{o}{=} \PY{n}{np}\PY{o}{.}\PY{n}{unique}\PY{p}{(}\PY{n}{dataframe}\PY{o}{.}\PY{n}{Outcome}\PY{p}{,} \PY{n}{return\PYZus{}counts}\PY{o}{=}\PY{k+kc}{True}\PY{p}{)}
\PY{k}{for} \PY{n}{c}\PY{p}{,} \PY{n}{i} \PY{o+ow}{in} \PY{n+nb}{zip}\PY{p}{(}\PY{n}{classes}\PY{p}{,} \PY{n}{counts}\PY{p}{)}\PY{p}{:}
    \PY{n+nb}{print}\PY{p}{(}\PY{l+s+sa}{f}\PY{l+s+s2}{\PYZdq{}}\PY{l+s+s2}{Class }\PY{l+s+si}{\PYZob{}}\PY{n}{c}\PY{l+s+si}{\PYZcb{}}\PY{l+s+s2}{: }\PY{l+s+si}{\PYZob{}}\PY{n}{i}\PY{l+s+si}{\PYZcb{}}\PY{l+s+s2}{ samples}\PY{l+s+s2}{\PYZdq{}}\PY{p}{)}
\PY{n+nb}{print}\PY{p}{(}\PY{l+s+s2}{\PYZdq{}}\PY{l+s+s2}{Proporção da classe 1: }\PY{l+s+si}{\PYZob{}:.2f\PYZcb{}}\PY{l+s+s2}{\PYZpc{}}\PY{l+s+s2}{\PYZdq{}}\PY{o}{.}\PY{n}{format}\PY{p}{(}\PY{n}{dataframe}\PY{o}{.}\PY{n}{Outcome}\PY{o}{.}\PY{n}{mean}\PY{p}{(}\PY{p}{)} \PY{o}{*} \PY{l+m+mi}{100}\PY{p}{)}\PY{p}{)}
\end{Verbatim}
\end{tcolorbox}

    \begin{Verbatim}[commandchars=\\\{\}]
Class 0: 500 samples
Class 1: 268 samples
Proporção da classe 1: 34.90\%
    \end{Verbatim}

    Portanto, essas são todas as colunas do arquivo:

    \begin{tcolorbox}[breakable, size=fbox, boxrule=1pt, pad at break*=1mm,colback=cellbackground, colframe=cellborder]
\prompt{In}{incolor}{8}{\boxspacing}
\begin{Verbatim}[commandchars=\\\{\}]
\PY{k}{for} \PY{n}{i}\PY{p}{,} \PY{n}{c} \PY{o+ow}{in} \PY{n+nb}{enumerate}\PY{p}{(}\PY{n}{dataframe}\PY{o}{.}\PY{n}{columns}\PY{p}{)}\PY{p}{:}
    \PY{n+nb}{print}\PY{p}{(}\PY{l+s+s2}{\PYZdq{}}\PY{l+s+s2}{Column }\PY{l+s+si}{\PYZob{}\PYZcb{}}\PY{l+s+s2}{: }\PY{l+s+si}{\PYZob{}\PYZcb{}}\PY{l+s+s2}{\PYZdq{}}\PY{o}{.}\PY{n}{format}\PY{p}{(}\PY{n}{i}\PY{p}{,} \PY{n}{c}\PY{p}{)}\PY{p}{)}
\end{Verbatim}
\end{tcolorbox}

    \begin{Verbatim}[commandchars=\\\{\}]
Column 0: Pregnancies
Column 1: Glucose
Column 2: BloodPressure
Column 3: SkinThickness
Column 4: Insulin
Column 5: BMI
Column 6: DiabetesPedigreeFunction
Column 7: Age
Column 8: Outcome
    \end{Verbatim}

    Cada coluna será normalizada de maneira diferente. A coluna
``Pregnancies'' e ``Age'' seram normalizada por \texttt{min\ max}, pois
como essas colunas não tem outlliers (tem valores bem definidos) e não
tem distribuição normal, parece ser a escolha mais adequada. As colunas
que tem distribuição similar a normal serão normalizadas por média e
distãncia inter-quantil (nos percentis 10\% e 90\%), para evitar o
efeito de dados espúrios. A coluna DiabetesPedigreeFunction será
normalizada por transformação de histograma, já que a distribuição dessa
feature é um pouco diferente de uma distribuição mais convencional.

Também aqui já será dividido o dataset de holdout, que servirá de
validação. Serão utilizados 25\% dos dados para validação.

    \begin{tcolorbox}[breakable, size=fbox, boxrule=1pt, pad at break*=1mm,colback=cellbackground, colframe=cellborder]
\prompt{In}{incolor}{9}{\boxspacing}
\begin{Verbatim}[commandchars=\\\{\}]
\PY{n}{preprocessor} \PY{o}{=} \PY{n}{ColumnTransformer}\PY{p}{(}
    \PY{p}{[}
        \PY{p}{(}\PY{l+s+s2}{\PYZdq{}}\PY{l+s+s2}{min\PYZus{}max\PYZus{}1}\PY{l+s+s2}{\PYZdq{}}\PY{p}{,} \PY{n}{MinMaxScaler}\PY{p}{(}\PY{n}{feature\PYZus{}range}\PY{o}{=}\PY{p}{(}\PY{l+m+mi}{0}\PY{p}{,} \PY{l+m+mi}{1}\PY{p}{)}\PY{p}{)}\PY{p}{,} \PY{p}{[}\PY{l+m+mi}{0}\PY{p}{]}\PY{p}{)}\PY{p}{,}
        \PY{p}{(}\PY{l+s+s2}{\PYZdq{}}\PY{l+s+s2}{robust\PYZus{}scaler\PYZus{}1}\PY{l+s+s2}{\PYZdq{}}\PY{p}{,} \PY{n}{RobustScaler}\PY{p}{(}\PY{n}{quantile\PYZus{}range}\PY{o}{=}\PY{p}{(}\PY{l+m+mf}{10.0}\PY{p}{,} \PY{l+m+mf}{90.0}\PY{p}{)}\PY{p}{)}\PY{p}{,} \PY{p}{[}\PY{l+m+mi}{1}\PY{p}{]}\PY{p}{)}\PY{p}{,}
        \PY{p}{(}\PY{l+s+s2}{\PYZdq{}}\PY{l+s+s2}{robust\PYZus{}scaler\PYZus{}2}\PY{l+s+s2}{\PYZdq{}}\PY{p}{,} \PY{n}{RobustScaler}\PY{p}{(}\PY{n}{quantile\PYZus{}range}\PY{o}{=}\PY{p}{(}\PY{l+m+mf}{10.0}\PY{p}{,} \PY{l+m+mf}{90.0}\PY{p}{)}\PY{p}{)}\PY{p}{,} \PY{p}{[}\PY{l+m+mi}{2}\PY{p}{]}\PY{p}{)}\PY{p}{,}
        \PY{p}{(}\PY{l+s+s2}{\PYZdq{}}\PY{l+s+s2}{robust\PYZus{}scaler\PYZus{}3}\PY{l+s+s2}{\PYZdq{}}\PY{p}{,} \PY{n}{RobustScaler}\PY{p}{(}\PY{n}{quantile\PYZus{}range}\PY{o}{=}\PY{p}{(}\PY{l+m+mf}{10.0}\PY{p}{,} \PY{l+m+mf}{90.0}\PY{p}{)}\PY{p}{)}\PY{p}{,} \PY{p}{[}\PY{l+m+mi}{3}\PY{p}{]}\PY{p}{)}\PY{p}{,}
        \PY{p}{(}\PY{l+s+s2}{\PYZdq{}}\PY{l+s+s2}{robust\PYZus{}scaler\PYZus{}4}\PY{l+s+s2}{\PYZdq{}}\PY{p}{,} \PY{n}{RobustScaler}\PY{p}{(}\PY{n}{quantile\PYZus{}range}\PY{o}{=}\PY{p}{(}\PY{l+m+mf}{10.0}\PY{p}{,} \PY{l+m+mf}{90.0}\PY{p}{)}\PY{p}{)}\PY{p}{,} \PY{p}{[}\PY{l+m+mi}{4}\PY{p}{]}\PY{p}{)}\PY{p}{,}
        \PY{p}{(}\PY{l+s+s2}{\PYZdq{}}\PY{l+s+s2}{robust\PYZus{}scaler\PYZus{}5}\PY{l+s+s2}{\PYZdq{}}\PY{p}{,} \PY{n}{RobustScaler}\PY{p}{(}\PY{n}{quantile\PYZus{}range}\PY{o}{=}\PY{p}{(}\PY{l+m+mf}{10.0}\PY{p}{,} \PY{l+m+mf}{90.0}\PY{p}{)}\PY{p}{)}\PY{p}{,} \PY{p}{[}\PY{l+m+mi}{5}\PY{p}{]}\PY{p}{)}\PY{p}{,}
        \PY{p}{(}\PY{l+s+s2}{\PYZdq{}}\PY{l+s+s2}{quantile\PYZus{}transform}\PY{l+s+s2}{\PYZdq{}}\PY{p}{,} \PY{n}{QuantileTransformer}\PY{p}{(}\PY{n}{n\PYZus{}quantiles}\PY{o}{=}\PY{l+m+mi}{50}\PY{p}{,} \PY{n}{output\PYZus{}distribution}\PY{o}{=}\PY{l+s+s1}{\PYZsq{}}\PY{l+s+s1}{uniform}\PY{l+s+s1}{\PYZsq{}}\PY{p}{)}\PY{p}{,} \PY{p}{[}\PY{l+m+mi}{6}\PY{p}{]}\PY{p}{)}\PY{p}{,}
        \PY{p}{(}\PY{l+s+s2}{\PYZdq{}}\PY{l+s+s2}{min\PYZus{}max\PYZus{}2}\PY{l+s+s2}{\PYZdq{}}\PY{p}{,} \PY{n}{MinMaxScaler}\PY{p}{(}\PY{n}{feature\PYZus{}range}\PY{o}{=}\PY{p}{(}\PY{o}{\PYZhy{}}\PY{l+m+mi}{1}\PY{p}{,} \PY{l+m+mi}{1}\PY{p}{)}\PY{p}{)}\PY{p}{,} \PY{p}{[}\PY{l+m+mi}{7}\PY{p}{]}\PY{p}{)}\PY{p}{,}
    \PY{p}{]}
\PY{p}{)}

\PY{n}{features} \PY{o}{=} \PY{n}{dataframe}\PY{o}{.}\PY{n}{iloc}\PY{p}{[}\PY{p}{:}\PY{p}{,} \PY{p}{:}\PY{o}{\PYZhy{}}\PY{l+m+mi}{1}\PY{p}{]}\PY{o}{.}\PY{n}{values}
\PY{n}{labels} \PY{o}{=} \PY{n}{dataframe}\PY{o}{.}\PY{n}{iloc}\PY{p}{[}\PY{p}{:}\PY{p}{,} \PY{p}{[}\PY{l+m+mi}{8}\PY{p}{]}\PY{p}{]}\PY{o}{.}\PY{n}{values}
\PY{n}{x\PYZus{}trainval}\PY{p}{,} \PY{n}{x\PYZus{}test}\PY{p}{,} \PY{n}{y\PYZus{}trainval}\PY{p}{,} \PY{n}{y\PYZus{}test} \PY{o}{=} \PY{n}{train\PYZus{}test\PYZus{}split}\PY{p}{(}\PY{n}{features}\PY{p}{,} \PY{n}{labels}\PY{p}{,} \PY{n}{test\PYZus{}size}\PY{o}{=}\PY{l+m+mf}{0.25}\PY{p}{,} \PY{n}{stratify}\PY{o}{=}\PY{n}{labels}\PY{p}{)}
\end{Verbatim}
\end{tcolorbox}

    Após a transformação, as distribuições são:

    \begin{tcolorbox}[breakable, size=fbox, boxrule=1pt, pad at break*=1mm,colback=cellbackground, colframe=cellborder]
\prompt{In}{incolor}{10}{\boxspacing}
\begin{Verbatim}[commandchars=\\\{\}]
\PY{n}{features\PYZus{}n} \PY{o}{=} \PY{n}{preprocessor}\PY{o}{.}\PY{n}{fit\PYZus{}transform}\PY{p}{(}\PY{n}{x\PYZus{}trainval}\PY{p}{)}
\PY{n}{plt}\PY{o}{.}\PY{n}{figure}\PY{p}{(}\PY{n}{figsize}\PY{o}{=}\PY{p}{(}\PY{l+m+mi}{12}\PY{p}{,} \PY{l+m+mi}{8}\PY{p}{)}\PY{p}{)}
\PY{n}{pd}\PY{o}{.}\PY{n}{DataFrame}\PY{p}{(}\PY{n}{features\PYZus{}n}\PY{p}{,} \PY{n}{columns}\PY{o}{=}\PY{n}{dataframe}\PY{o}{.}\PY{n}{columns}\PY{p}{[}\PY{p}{:}\PY{o}{\PYZhy{}}\PY{l+m+mi}{1}\PY{p}{]}\PY{p}{)}\PY{o}{.}\PY{n}{hist}\PY{p}{(}\PY{n}{bins}\PY{o}{=}\PY{l+m+mi}{10}\PY{p}{,} \PY{n}{ax}\PY{o}{=}\PY{n}{plt}\PY{o}{.}\PY{n}{gca}\PY{p}{(}\PY{p}{)}\PY{p}{)}\PY{p}{;}
\PY{n}{plt}\PY{o}{.}\PY{n}{tight\PYZus{}layout}\PY{p}{(}\PY{p}{)}
\end{Verbatim}
\end{tcolorbox}

    \begin{center}
    \adjustimage{max size={0.9\linewidth}{0.9\paperheight}}{EFC3_Pt1_files/EFC3_Pt1_20_0.png}
    \end{center}
    { \hspace*{\fill} \\}
    
    \hypertarget{treinamento}{%
\subsection{Treinamento}\label{treinamento}}

    Neste passo várias MLP's de uma camada escondida serão treinadas,
utilizando o protocolo de validação cruzada, para se encontrar o melhor
número de neurônios da camada escondida. A validação cruzada será feita
utilizando 10 folds. Desta maneira pode-se achar o melhor número de
neurônios para este problema, para posteriormente treinar o modelo
final.

Serão treinadas 100 redes com 2, 4, 6, \ldots, 200 neurônios na camada
escondida. Após os treinamentos, a rede com melhor acurácia balanceada
na validação cruzada (utilizando apenas dados de treino) será escolhida,
e será retreinada em todos dados de treino.

As redes serão treinadas utilizando o otimizar Adam, e o \texttt{alpha}
(que é o parametro de regularização l2 para esta classe do sklearn) será
setado para zero, assim cada modelo poderá se ajustar sem restrições
para que se possa ver com mais clareza o efeito do aumento do número de
neurônios.

    \begin{tcolorbox}[breakable, size=fbox, boxrule=1pt, pad at break*=1mm,colback=cellbackground, colframe=cellborder]
\prompt{In}{incolor}{11}{\boxspacing}
\begin{Verbatim}[commandchars=\\\{\}]
\PY{n}{METRICS} \PY{o}{=} \PY{p}{[}\PY{n}{f1\PYZus{}score}\PY{p}{,} \PY{n}{precision\PYZus{}score}\PY{p}{,} \PY{n}{recall\PYZus{}score}\PY{p}{,} \PY{n}{balanced\PYZus{}accuracy\PYZus{}score}\PY{p}{]}

\PY{k}{def} \PY{n+nf}{evaluate}\PY{p}{(}\PY{n}{y\PYZus{}target}\PY{p}{,} \PY{n}{y\PYZus{}pred}\PY{p}{,} \PY{n}{metrics}\PY{o}{=}\PY{n}{METRICS}\PY{p}{)}\PY{p}{:}
    \PY{n}{results} \PY{o}{=} \PY{p}{\PYZob{}}\PY{p}{\PYZcb{}}
    \PY{k}{for} \PY{n}{metric} \PY{o+ow}{in} \PY{n}{metrics}\PY{p}{:}
        \PY{n}{name} \PY{o}{=} \PY{n}{metric}\PY{o}{.}\PY{n+nv+vm}{\PYZus{}\PYZus{}name\PYZus{}\PYZus{}}
        \PY{n}{results}\PY{p}{[}\PY{n}{name}\PY{p}{]} \PY{o}{=} \PY{n}{metric}\PY{p}{(}\PY{n}{y\PYZus{}target}\PY{p}{,} \PY{n}{y\PYZus{}pred}\PY{p}{)}
    \PY{k}{return} \PY{n}{results}

\PY{k}{def} \PY{n+nf}{train\PYZus{}model}\PY{p}{(}\PY{n}{model\PYZus{}fn}\PY{p}{,} \PY{n}{kwargs}\PY{p}{,} \PY{n}{X}\PY{p}{,} \PY{n}{Y}\PY{p}{,} \PY{n}{metrics}\PY{o}{=}\PY{n}{METRICS}\PY{p}{,} \PY{n}{n\PYZus{}folds}\PY{o}{=}\PY{l+m+mi}{10}\PY{p}{)}\PY{p}{:}
    \PY{l+s+sd}{\PYZdq{}\PYZdq{}\PYZdq{} Train the model generated by `model\PYZus{}fn` using a cross validation protocol.}
\PY{l+s+sd}{    Returns metrics calculated for each fold, for the train and validation sets.}
\PY{l+s+sd}{    \PYZdq{}\PYZdq{}\PYZdq{}}
    \PY{n}{model} \PY{o}{=} \PY{n}{model\PYZus{}fn}\PY{p}{(}\PY{o}{*}\PY{o}{*} \PY{n}{kwargs}\PY{p}{)}
    \PY{n}{skf} \PY{o}{=} \PY{n}{StratifiedKFold}\PY{p}{(}\PY{n}{n\PYZus{}splits}\PY{o}{=}\PY{n}{n\PYZus{}folds}\PY{p}{,} \PY{n}{random\PYZus{}state}\PY{o}{=}\PY{n}{SEED}\PY{p}{)}
    
    \PY{n}{train\PYZus{}results} \PY{o}{=} \PY{p}{\PYZob{}}\PY{n}{metric}\PY{o}{.}\PY{n+nv+vm}{\PYZus{}\PYZus{}name\PYZus{}\PYZus{}}\PY{p}{:} \PY{p}{[}\PY{p}{]} \PY{k}{for} \PY{n}{metric} \PY{o+ow}{in} \PY{n}{metrics}\PY{p}{\PYZcb{}}
    \PY{n}{val\PYZus{}results} \PY{o}{=} \PY{p}{\PYZob{}}\PY{n}{metric}\PY{o}{.}\PY{n+nv+vm}{\PYZus{}\PYZus{}name\PYZus{}\PYZus{}}\PY{p}{:} \PY{p}{[}\PY{p}{]} \PY{k}{for} \PY{n}{metric} \PY{o+ow}{in} \PY{n}{metrics}\PY{p}{\PYZcb{}}
    \PY{k}{for} \PY{n}{train\PYZus{}idx}\PY{p}{,} \PY{n}{val\PYZus{}idx} \PY{o+ow}{in} \PY{n}{skf}\PY{o}{.}\PY{n}{split}\PY{p}{(}\PY{n}{X}\PY{p}{,} \PY{n}{Y}\PY{p}{)}\PY{p}{:}
        \PY{n}{train\PYZus{}features}\PY{p}{,} \PY{n}{train\PYZus{}labels} \PY{o}{=} \PY{n}{X}\PY{p}{[}\PY{n}{train\PYZus{}idx}\PY{p}{]}\PY{p}{,} \PY{n}{Y}\PY{p}{[}\PY{n}{train\PYZus{}idx}\PY{p}{]}
        \PY{n}{val\PYZus{}features}\PY{p}{,} \PY{n}{val\PYZus{}labels} \PY{o}{=} \PY{n}{X}\PY{p}{[}\PY{n}{val\PYZus{}idx}\PY{p}{]}\PY{p}{,} \PY{n}{Y}\PY{p}{[}\PY{n}{val\PYZus{}idx}\PY{p}{]}
        
        \PY{n}{train\PYZus{}features} \PY{o}{=} \PY{n}{preprocessor}\PY{o}{.}\PY{n}{fit\PYZus{}transform}\PY{p}{(}\PY{n}{train\PYZus{}features}\PY{p}{)}
        \PY{n}{val\PYZus{}features} \PY{o}{=} \PY{n}{preprocessor}\PY{o}{.}\PY{n}{transform}\PY{p}{(}\PY{n}{val\PYZus{}features}\PY{p}{)}
        
        \PY{n}{model}\PY{o}{.}\PY{n}{fit}\PY{p}{(}\PY{n}{train\PYZus{}features}\PY{p}{,} \PY{n}{train\PYZus{}labels}\PY{p}{)}
        \PY{n}{train\PYZus{}pred} \PY{o}{=} \PY{n}{model}\PY{o}{.}\PY{n}{predict}\PY{p}{(}\PY{n}{train\PYZus{}features}\PY{p}{)}
        \PY{n}{val\PYZus{}pred} \PY{o}{=} \PY{n}{model}\PY{o}{.}\PY{n}{predict}\PY{p}{(}\PY{n}{val\PYZus{}features}\PY{p}{)}
        
        \PY{k}{for} \PY{n}{metric} \PY{o+ow}{in} \PY{n}{metrics}\PY{p}{:}
            \PY{n}{name} \PY{o}{=} \PY{n}{metric}\PY{o}{.}\PY{n+nv+vm}{\PYZus{}\PYZus{}name\PYZus{}\PYZus{}}
            \PY{n}{train\PYZus{}results}\PY{p}{[}\PY{n}{name}\PY{p}{]}\PY{o}{.}\PY{n}{append}\PY{p}{(}
                \PY{n}{metric}\PY{p}{(}\PY{n}{train\PYZus{}labels}\PY{p}{,} \PY{n}{train\PYZus{}pred}\PY{p}{)}
            \PY{p}{)}
            \PY{n}{val\PYZus{}results}\PY{p}{[}\PY{n}{name}\PY{p}{]}\PY{o}{.}\PY{n}{append}\PY{p}{(}
                \PY{n}{metric}\PY{p}{(}\PY{n}{val\PYZus{}labels}\PY{p}{,} \PY{n}{val\PYZus{}pred}\PY{p}{)}
            \PY{p}{)}
    \PY{k}{return} \PY{n}{train\PYZus{}results}\PY{p}{,} \PY{n}{val\PYZus{}results}

\PY{n}{train\PYZus{}results}\PY{p}{,} \PY{n}{val\PYZus{}results} \PY{o}{=} \PY{p}{[}\PY{p}{]}\PY{p}{,} \PY{p}{[}\PY{p}{]}
\PY{n}{n\PYZus{}neurons\PYZus{}arr} \PY{o}{=} \PY{n+nb}{list}\PY{p}{(}\PY{n+nb}{range}\PY{p}{(}\PY{l+m+mi}{2}\PY{p}{,} \PY{l+m+mi}{201}\PY{p}{,} \PY{l+m+mi}{2}\PY{p}{)}\PY{p}{)}

\PY{l+s+sd}{\PYZdq{}\PYZdq{}\PYZdq{}}
\PY{l+s+sd}{For each number of neurons, train a model using a 5\PYZhy{}fold cross validation, }
\PY{l+s+sd}{and stores the mean balanced accuracy score in the 5 folds.}
\PY{l+s+sd}{\PYZdq{}\PYZdq{}\PYZdq{}}
\PY{k}{for} \PY{n}{n\PYZus{}neurons} \PY{o+ow}{in} \PY{n}{tqdm}\PY{p}{(}\PY{n}{n\PYZus{}neurons\PYZus{}arr}\PY{p}{)}\PY{p}{:}
    \PY{n}{train\PYZus{}res}\PY{p}{,} \PY{n}{val\PYZus{}res} \PY{o}{=} \PY{n}{train\PYZus{}model}\PY{p}{(}
        \PY{n}{MLPClassifier}\PY{p}{,}
        \PY{p}{\PYZob{}}
            \PY{l+s+s2}{\PYZdq{}}\PY{l+s+s2}{hidden\PYZus{}layer\PYZus{}sizes}\PY{l+s+s2}{\PYZdq{}}\PY{p}{:} \PY{n}{n\PYZus{}neurons}\PY{p}{,}
            \PY{l+s+s2}{\PYZdq{}}\PY{l+s+s2}{learning\PYZus{}rate\PYZus{}init}\PY{l+s+s2}{\PYZdq{}}\PY{p}{:} \PY{l+m+mf}{0.001}\PY{p}{,}
            \PY{l+s+s2}{\PYZdq{}}\PY{l+s+s2}{max\PYZus{}iter}\PY{l+s+s2}{\PYZdq{}}\PY{p}{:} \PY{l+m+mi}{5000}\PY{p}{,}
            \PY{l+s+s2}{\PYZdq{}}\PY{l+s+s2}{random\PYZus{}state}\PY{l+s+s2}{\PYZdq{}}\PY{p}{:} \PY{n}{SEED}\PY{p}{,}
            \PY{l+s+s2}{\PYZdq{}}\PY{l+s+s2}{early\PYZus{}stopping}\PY{l+s+s2}{\PYZdq{}}\PY{p}{:} \PY{k+kc}{True}\PY{p}{,}
            \PY{l+s+s2}{\PYZdq{}}\PY{l+s+s2}{n\PYZus{}iter\PYZus{}no\PYZus{}change}\PY{l+s+s2}{\PYZdq{}}\PY{p}{:} \PY{l+m+mi}{20}\PY{p}{,}
            \PY{l+s+s2}{\PYZdq{}}\PY{l+s+s2}{tol}\PY{l+s+s2}{\PYZdq{}}\PY{p}{:} \PY{l+m+mf}{1e\PYZhy{}3}\PY{p}{,}
            \PY{l+s+s2}{\PYZdq{}}\PY{l+s+s2}{alpha}\PY{l+s+s2}{\PYZdq{}}\PY{p}{:} \PY{l+m+mi}{0}
        \PY{p}{\PYZcb{}}\PY{p}{,}
        \PY{n}{x\PYZus{}trainval}\PY{p}{,}
        \PY{n}{y\PYZus{}trainval}
    \PY{p}{)}
    \PY{n}{train\PYZus{}results}\PY{o}{.}\PY{n}{append}\PY{p}{(}
        \PY{n}{np}\PY{o}{.}\PY{n}{mean}\PY{p}{(}\PY{n}{train\PYZus{}res}\PY{p}{[}\PY{l+s+s2}{\PYZdq{}}\PY{l+s+s2}{balanced\PYZus{}accuracy\PYZus{}score}\PY{l+s+s2}{\PYZdq{}}\PY{p}{]}\PY{p}{)}
    \PY{p}{)}
    \PY{n}{val\PYZus{}results}\PY{o}{.}\PY{n}{append}\PY{p}{(}
        \PY{n}{np}\PY{o}{.}\PY{n}{mean}\PY{p}{(}\PY{n}{val\PYZus{}res}\PY{p}{[}\PY{l+s+s2}{\PYZdq{}}\PY{l+s+s2}{balanced\PYZus{}accuracy\PYZus{}score}\PY{l+s+s2}{\PYZdq{}}\PY{p}{]}\PY{p}{)}
    \PY{p}{)}
\end{Verbatim}
\end{tcolorbox}

    \begin{Verbatim}[commandchars=\\\{\}]
100\%|███████████████████████████████████████████████████████████████████████████
█████| 100/100 [03:43<00:00,  2.23s/it]
    \end{Verbatim}

    \begin{tcolorbox}[breakable, size=fbox, boxrule=1pt, pad at break*=1mm,colback=cellbackground, colframe=cellborder]
\prompt{In}{incolor}{12}{\boxspacing}
\begin{Verbatim}[commandchars=\\\{\}]
\PY{n}{plt}\PY{o}{.}\PY{n}{figure}\PY{p}{(}\PY{n}{figsize}\PY{o}{=}\PY{p}{(}\PY{l+m+mi}{12}\PY{p}{,} \PY{l+m+mi}{8}\PY{p}{)}\PY{p}{)}
\PY{n}{plt}\PY{o}{.}\PY{n}{plot}\PY{p}{(}\PY{n}{n\PYZus{}neurons\PYZus{}arr}\PY{p}{,} \PY{n}{train\PYZus{}results}\PY{p}{,} \PY{n}{label}\PY{o}{=}\PY{l+s+s2}{\PYZdq{}}\PY{l+s+s2}{Treino}\PY{l+s+s2}{\PYZdq{}}\PY{p}{)}
\PY{n}{plt}\PY{o}{.}\PY{n}{plot}\PY{p}{(}\PY{n}{n\PYZus{}neurons\PYZus{}arr}\PY{p}{,} \PY{n}{val\PYZus{}results}\PY{p}{,} \PY{n}{label}\PY{o}{=}\PY{l+s+s2}{\PYZdq{}}\PY{l+s+s2}{Validação}\PY{l+s+s2}{\PYZdq{}}\PY{p}{)}
\PY{n}{plt}\PY{o}{.}\PY{n}{legend}\PY{p}{(}\PY{p}{)}
\PY{n}{plt}\PY{o}{.}\PY{n}{grid}\PY{p}{(}\PY{p}{)}
\PY{n}{plt}\PY{o}{.}\PY{n}{xlabel}\PY{p}{(}\PY{l+s+s2}{\PYZdq{}}\PY{l+s+s2}{Número de neurônios}\PY{l+s+s2}{\PYZdq{}}\PY{p}{)}
\PY{n}{plt}\PY{o}{.}\PY{n}{ylabel}\PY{p}{(}\PY{l+s+s2}{\PYZdq{}}\PY{l+s+s2}{Acurácia Balanceada}\PY{l+s+s2}{\PYZdq{}}\PY{p}{)}
\end{Verbatim}
\end{tcolorbox}

            \begin{tcolorbox}[breakable, size=fbox, boxrule=.5pt, pad at break*=1mm, opacityfill=0]
\prompt{Out}{outcolor}{12}{\boxspacing}
\begin{Verbatim}[commandchars=\\\{\}]
Text(0, 0.5, 'Acurácia Balanceada')
\end{Verbatim}
\end{tcolorbox}
        
    \begin{center}
    \adjustimage{max size={0.9\linewidth}{0.9\paperheight}}{EFC3_Pt1_files/EFC3_Pt1_24_1.png}
    \end{center}
    { \hspace*{\fill} \\}
    
    Verifica-se que o modelo tem performance semelhante na validação
independente do número de neurônios utilizados aqui. Se for ocorrer
overfitting devido ao tamanho do modelo, talvez ele ocorra com um número
superior de neurônios.

    Encontrando-se o número ótimo de neurônios para o modelo, agora ele pode
ser instanciado.

Obs: Aqui será utilizado outra biblioteca para instanciação do modelo
pela facilidade que ela dá em retornar os erros de treino e validação. A
sua performance não será afetada.

    \begin{tcolorbox}[breakable, size=fbox, boxrule=1pt, pad at break*=1mm,colback=cellbackground, colframe=cellborder]
\prompt{In}{incolor}{13}{\boxspacing}
\begin{Verbatim}[commandchars=\\\{\}]
\PY{n}{best\PYZus{}n\PYZus{}neurons} \PY{o}{=} \PY{n}{n\PYZus{}neurons\PYZus{}arr}\PY{p}{[}\PY{n}{np}\PY{o}{.}\PY{n}{argmax}\PY{p}{(}\PY{n}{val\PYZus{}results}\PY{p}{)}\PY{p}{]}
\PY{n+nb}{print}\PY{p}{(}\PY{l+s+s2}{\PYZdq{}}\PY{l+s+s2}{O Número ótimo de neurônios é}\PY{l+s+s2}{\PYZdq{}}\PY{p}{,} \PY{n}{best\PYZus{}n\PYZus{}neurons}\PY{p}{)}
\PY{n}{model} \PY{o}{=} \PY{n}{tf}\PY{o}{.}\PY{n}{keras}\PY{o}{.}\PY{n}{Sequential}\PY{p}{(}
    \PY{p}{[}
        \PY{n}{tf}\PY{o}{.}\PY{n}{keras}\PY{o}{.}\PY{n}{layers}\PY{o}{.}\PY{n}{InputLayer}\PY{p}{(}\PY{n}{input\PYZus{}shape}\PY{o}{=}\PY{l+m+mi}{8}\PY{p}{)}\PY{p}{,}
        \PY{n}{tf}\PY{o}{.}\PY{n}{keras}\PY{o}{.}\PY{n}{layers}\PY{o}{.}\PY{n}{Dense}\PY{p}{(}\PY{n}{best\PYZus{}n\PYZus{}neurons}\PY{p}{)}\PY{p}{,}
        \PY{n}{tf}\PY{o}{.}\PY{n}{keras}\PY{o}{.}\PY{n}{layers}\PY{o}{.}\PY{n}{Dense}\PY{p}{(}\PY{l+m+mi}{1}\PY{p}{,} \PY{n}{activation}\PY{o}{=}\PY{l+s+s2}{\PYZdq{}}\PY{l+s+s2}{sigmoid}\PY{l+s+s2}{\PYZdq{}}\PY{p}{)}\PY{p}{,}
    \PY{p}{]}
\PY{p}{)}

\PY{n}{model}\PY{o}{.}\PY{n}{compile}\PY{p}{(}
    \PY{n}{optimizer}\PY{o}{=}\PY{n}{tf}\PY{o}{.}\PY{n}{keras}\PY{o}{.}\PY{n}{optimizers}\PY{o}{.}\PY{n}{Adam}\PY{p}{(}\PY{n}{lr}\PY{o}{=}\PY{l+m+mf}{1e\PYZhy{}3}\PY{p}{)}\PY{p}{,}
    \PY{n}{loss}\PY{o}{=}\PY{l+s+s2}{\PYZdq{}}\PY{l+s+s2}{binary\PYZus{}crossentropy}\PY{l+s+s2}{\PYZdq{}}\PY{p}{,}
    \PY{n}{loss\PYZus{}weights}\PY{o}{=}\PY{k+kc}{None}\PY{p}{,}
    \PY{n}{metrics}\PY{o}{=}\PY{p}{[}\PY{l+s+s2}{\PYZdq{}}\PY{l+s+s2}{accuracy}\PY{l+s+s2}{\PYZdq{}}\PY{p}{]}
\PY{p}{)}
\PY{n}{model}\PY{o}{.}\PY{n}{summary}\PY{p}{(}\PY{p}{)}
\end{Verbatim}
\end{tcolorbox}

    \begin{Verbatim}[commandchars=\\\{\}]
O Número ótimo de neurônios é 146
Model: "sequential"
\_\_\_\_\_\_\_\_\_\_\_\_\_\_\_\_\_\_\_\_\_\_\_\_\_\_\_\_\_\_\_\_\_\_\_\_\_\_\_\_\_\_\_\_\_\_\_\_\_\_\_\_\_\_\_\_\_\_\_\_\_\_\_\_\_
Layer (type)                 Output Shape              Param \#
=================================================================
dense (Dense)                (None, 146)               1314
\_\_\_\_\_\_\_\_\_\_\_\_\_\_\_\_\_\_\_\_\_\_\_\_\_\_\_\_\_\_\_\_\_\_\_\_\_\_\_\_\_\_\_\_\_\_\_\_\_\_\_\_\_\_\_\_\_\_\_\_\_\_\_\_\_
dense\_1 (Dense)              (None, 1)                 147
=================================================================
Total params: 1,461
Trainable params: 1,461
Non-trainable params: 0
\_\_\_\_\_\_\_\_\_\_\_\_\_\_\_\_\_\_\_\_\_\_\_\_\_\_\_\_\_\_\_\_\_\_\_\_\_\_\_\_\_\_\_\_\_\_\_\_\_\_\_\_\_\_\_\_\_\_\_\_\_\_\_\_\_
    \end{Verbatim}

    \begin{tcolorbox}[breakable, size=fbox, boxrule=1pt, pad at break*=1mm,colback=cellbackground, colframe=cellborder]
\prompt{In}{incolor}{14}{\boxspacing}
\begin{Verbatim}[commandchars=\\\{\}]
\PY{n}{X\PYZus{}train} \PY{o}{=} \PY{n}{preprocessor}\PY{o}{.}\PY{n}{fit\PYZus{}transform}\PY{p}{(}\PY{n}{x\PYZus{}trainval}\PY{p}{)}
\PY{n}{Y\PYZus{}train} \PY{o}{=} \PY{n}{y\PYZus{}trainval}
\PY{n}{X\PYZus{}val} \PY{o}{=} \PY{n}{preprocessor}\PY{o}{.}\PY{n}{transform}\PY{p}{(}\PY{n}{x\PYZus{}test}\PY{p}{)}
\PY{n}{Y\PYZus{}val} \PY{o}{=} \PY{n}{y\PYZus{}test}

\PY{n}{callbacks} \PY{o}{=} \PY{n}{EarlyStopping}\PY{p}{(}
    \PY{n}{monitor}\PY{o}{=}\PY{l+s+s2}{\PYZdq{}}\PY{l+s+s2}{val\PYZus{}loss}\PY{l+s+s2}{\PYZdq{}}\PY{p}{,}
    \PY{n}{patience}\PY{o}{=}\PY{l+m+mi}{20}\PY{p}{,}
    \PY{n}{min\PYZus{}delta}\PY{o}{=}\PY{l+m+mf}{1e\PYZhy{}3}\PY{p}{,}
    \PY{n}{restore\PYZus{}best\PYZus{}weights}\PY{o}{=}\PY{k+kc}{True}
\PY{p}{)}

\PY{n}{history} \PY{o}{=} \PY{n}{model}\PY{o}{.}\PY{n}{fit}\PY{p}{(}\PY{n}{X\PYZus{}train}\PY{p}{,} \PY{n}{Y\PYZus{}train}\PY{p}{,} \PY{n}{validation\PYZus{}data}\PY{o}{=}\PY{p}{(}\PY{n}{X\PYZus{}val}\PY{p}{,} \PY{n}{Y\PYZus{}val}\PY{p}{)}\PY{p}{,} \PY{n}{callbacks}\PY{o}{=}\PY{n}{callbacks}\PY{p}{,} \PY{n}{epochs}\PY{o}{=}\PY{l+m+mi}{5000}\PY{p}{,} \PY{n}{batch\PYZus{}size}\PY{o}{=}\PY{l+m+mi}{32}\PY{p}{,} \PY{n}{verbose}\PY{o}{=}\PY{l+m+mi}{0}\PY{p}{)}
\end{Verbatim}
\end{tcolorbox}

    \begin{tcolorbox}[breakable, size=fbox, boxrule=1pt, pad at break*=1mm,colback=cellbackground, colframe=cellborder]
\prompt{In}{incolor}{15}{\boxspacing}
\begin{Verbatim}[commandchars=\\\{\}]
\PY{n}{pred} \PY{o}{=} \PY{p}{(}\PY{n}{model}\PY{o}{.}\PY{n}{predict}\PY{p}{(}\PY{n}{X\PYZus{}val}\PY{p}{)} \PY{o}{\PYZgt{}} \PY{l+m+mf}{0.5}\PY{p}{)}\PY{o}{.}\PY{n}{astype}\PY{p}{(}\PY{n}{np}\PY{o}{.}\PY{n}{int}\PY{p}{)}

\PY{n}{bas} \PY{o}{=} \PY{n}{balanced\PYZus{}accuracy\PYZus{}score}\PY{p}{(}\PY{n}{Y\PYZus{}val}\PY{p}{,} \PY{n}{pred}\PY{p}{)}
\PY{n}{acs} \PY{o}{=} \PY{n}{accuracy\PYZus{}score}\PY{p}{(}\PY{n}{Y\PYZus{}val}\PY{p}{,} \PY{n}{pred}\PY{p}{)}
\PY{n+nb}{print}\PY{p}{(}\PY{l+s+s2}{\PYZdq{}}\PY{l+s+s2}{Matriz de confusão}\PY{l+s+s2}{\PYZdq{}}\PY{p}{)}
\PY{n+nb}{print}\PY{p}{(}\PY{n}{confusion\PYZus{}matrix}\PY{p}{(}\PY{n}{Y\PYZus{}val}\PY{p}{,} \PY{n}{pred}\PY{p}{)}\PY{p}{)}
\PY{n+nb}{print}\PY{p}{(}\PY{p}{)}
\PY{n+nb}{print}\PY{p}{(}\PY{l+s+s2}{\PYZdq{}}\PY{l+s+s2}{Estatísticas da predição}\PY{l+s+s2}{\PYZdq{}}\PY{p}{)}
\PY{n+nb}{print}\PY{p}{(}\PY{n}{classification\PYZus{}report}\PY{p}{(}\PY{n}{Y\PYZus{}val}\PY{p}{,} \PY{n}{pred}\PY{p}{)}\PY{p}{)}

\PY{n+nb}{print}\PY{p}{(}\PY{l+s+s2}{\PYZdq{}}\PY{l+s+s2}{Acurácia: }\PY{l+s+si}{\PYZob{}:.2f\PYZcb{}}\PY{l+s+s2}{\PYZpc{}}\PY{l+s+s2}{\PYZdq{}}\PY{o}{.}\PY{n}{format}\PY{p}{(}\PY{n}{acs} \PY{o}{*} \PY{l+m+mi}{100}\PY{p}{)}\PY{p}{)}
\PY{n+nb}{print}\PY{p}{(}\PY{l+s+s2}{\PYZdq{}}\PY{l+s+s2}{Acurácia balanceada: }\PY{l+s+si}{\PYZob{}:.2f\PYZcb{}}\PY{l+s+s2}{\PYZpc{}}\PY{l+s+s2}{\PYZdq{}}\PY{o}{.}\PY{n}{format}\PY{p}{(}\PY{n}{bas} \PY{o}{*} \PY{l+m+mi}{100}\PY{p}{)}\PY{p}{)}
\end{Verbatim}
\end{tcolorbox}

    \begin{Verbatim}[commandchars=\\\{\}]
Matriz de confusão
[[108  17]
 [ 27  40]]

Estatísticas da predição
              precision    recall  f1-score   support

           0       0.80      0.86      0.83       125
           1       0.70      0.60      0.65        67

    accuracy                           0.77       192
   macro avg       0.75      0.73      0.74       192
weighted avg       0.77      0.77      0.77       192

Acurácia: 77.08\%
Acurácia balanceada: 73.05\%
    \end{Verbatim}

    \begin{tcolorbox}[breakable, size=fbox, boxrule=1pt, pad at break*=1mm,colback=cellbackground, colframe=cellborder]
\prompt{In}{incolor}{16}{\boxspacing}
\begin{Verbatim}[commandchars=\\\{\}]
\PY{n}{plt}\PY{o}{.}\PY{n}{plot}\PY{p}{(}\PY{n}{history}\PY{o}{.}\PY{n}{history}\PY{p}{[}\PY{l+s+s2}{\PYZdq{}}\PY{l+s+s2}{loss}\PY{l+s+s2}{\PYZdq{}}\PY{p}{]}\PY{p}{,} \PY{n}{label}\PY{o}{=}\PY{l+s+s2}{\PYZdq{}}\PY{l+s+s2}{Erro de treino}\PY{l+s+s2}{\PYZdq{}}\PY{p}{)}
\PY{n}{plt}\PY{o}{.}\PY{n}{plot}\PY{p}{(}\PY{n}{history}\PY{o}{.}\PY{n}{history}\PY{p}{[}\PY{l+s+s2}{\PYZdq{}}\PY{l+s+s2}{val\PYZus{}loss}\PY{l+s+s2}{\PYZdq{}}\PY{p}{]}\PY{p}{,} \PY{n}{label}\PY{o}{=}\PY{l+s+s2}{\PYZdq{}}\PY{l+s+s2}{Erro de validação}\PY{l+s+s2}{\PYZdq{}}\PY{p}{)}
\PY{n}{plt}\PY{o}{.}\PY{n}{xlabel}\PY{p}{(}\PY{l+s+s2}{\PYZdq{}}\PY{l+s+s2}{Época}\PY{l+s+s2}{\PYZdq{}}\PY{p}{)}
\PY{n}{plt}\PY{o}{.}\PY{n}{ylabel}\PY{p}{(}\PY{l+s+s2}{\PYZdq{}}\PY{l+s+s2}{Erro}\PY{l+s+s2}{\PYZdq{}}\PY{p}{)}
\PY{n}{plt}\PY{o}{.}\PY{n}{legend}\PY{p}{(}\PY{p}{)}
\end{Verbatim}
\end{tcolorbox}

            \begin{tcolorbox}[breakable, size=fbox, boxrule=.5pt, pad at break*=1mm, opacityfill=0]
\prompt{Out}{outcolor}{16}{\boxspacing}
\begin{Verbatim}[commandchars=\\\{\}]
<matplotlib.legend.Legend at 0x2755c39b188>
\end{Verbatim}
\end{tcolorbox}
        
    \begin{center}
    \adjustimage{max size={0.9\linewidth}{0.9\paperheight}}{EFC3_Pt1_files/EFC3_Pt1_30_1.png}
    \end{center}
    { \hspace*{\fill} \\}
    
    \hypertarget{conclusuxe3o}{%
\subsection{Conclusão}\label{conclusuxe3o}}

Percebe-se que quanto não necessariamente um número grande de neurônios
será o melhor para um modelo, devido ao overfitting. Entretanto neste
problema ainda não pode-se verificar o efeito do overfitting
completamente, já que no espaço de busca que se utilizou para os
neurônios o melhor valor encontrado foi alto. Também pode-se utilizar na
prática o protocolo de validação cruzada para se encontrar os melhores
hiperparâmetros deste modelo (neste caso, o número de neurônios),
verificando-se assim este método de otimização de parâmetros.


    % Add a bibliography block to the postdoc
    
    
    
\end{document}
